\documentclass[twocolumn,10pt]{article}
\usepackage[greek,francais,english]{babel}
\usepackage[T1]{fontenc}
\usepackage[utf8]{inputenc}
\usepackage[top=2cm, bottom=2cm, left=1cm, right=1cm]{geometry}
\setlength{\columnsep}{1cm}
\usepackage[usenames,dvipsnames]{color}
\usepackage{graphicx}
\newcommand\tet{\textgreek{tetraf'armakos}}
\begin{document}
\title{\tet%en cas de problème avec cette fonte, ne pas hesiter à enlever le \tet en le commentant avec un % suivit d'un passage à la ligne
: Vegetation, atmosphere\\ and flight parameters monitoring} 
\author{Clément Javerzac-Galy \and{Denis Savoie} \and{Zubair Iftikhar} }
\date{\today}
\maketitle

 
\begin{abstract}
 \begin{bfseries}
	Le cansat \tet \footnote{prononcez tetrapharmakos} a été réalisé l'équipe des \textit{Proton-thérapeutes} à l'occasion de la deuxième participation d'une équipe de l'Institut d'Optique \textit{Graduate School} au concours Cansat organisé par Planète-Sciences et le CNES. Ce mini-satellite embarque un module fait-maison permetant de mesurer le taux de végétation au sol. Il remplira en outre la mission de sondage atmosphérique. Lors de sa chute il mesurera certains paramètres de vol. Nous espérons que ces études complémentaires pourraient un jour servir à analyser des chances de vie sur une "Earth-like exoplanet".
 \end{bfseries}
\end{abstract}

\section{Introduction}
	



\begin{thebibliography}{9}
  % la biblio
\end{thebibliography}
\end{document}

